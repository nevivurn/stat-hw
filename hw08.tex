\documentclass{article}
\usepackage[a4paper]{geometry}
\usepackage{amsmath}
\usepackage{amssymb}
\usepackage{luatexko}
\usepackage{parskip}
\usepackage{svg}
\usepackage{enumitem}

\setlist[enumerate,1]{label=(\alph*)}

\setmainhangulfont{Nanum Myeongjo}
\setsanshangulfont{Nanum Gothic}

\title{과제8 : 9장 연습문제}
\author{성용운 2017-19937}
\date{}

\begin{document}

\maketitle

\section*{Problem 1}

\begin{align*}
	\sum^k \sum^n (y_{ij} - \bar{y}_{\cdot\cdot})^2
	&= \sum^k \sum^n (
		(y_{i\cdot} - \bar{y}_{\cdot\cdot}) +
		(y_{ij} - \bar{y}_{i\cdot})
	)^2 \\
	&= \sum^k \sum^n (
		(y_{i\cdot} - \bar{y}_{\cdot\cdot})^2 +
		(y_{ij} - \bar{y}_{i\cdot})^2 +
		2(y_{i\cdot} - \bar{y}_{\cdot\cdot})
		(y_{ij} - \bar{y}_{i\cdot})
	) \\
	&= \sum^k \sum^n (
		(y_{i\cdot} - \bar{y}_{\cdot\cdot})^2 +
		(y_{ij} - \bar{y}_{i\cdot})^2
	) +
		2 \sum^k (y_{i\cdot} - \bar{y}_{\cdot\cdot})
		\sum^n (y_{ij} - \bar{y}_{i\cdot})
	\\
	&= \sum^k \sum^n (
		(y_{i\cdot} - \bar{y}_{\cdot\cdot})^2 +
		(y_{ij} - \bar{y}_{i\cdot})^2
	) \\
	&=
	n \sum^k (y_{i\cdot} - \bar{y}_{\cdot\cdot})^2 +
	\sum^k \sum^n (y_{ij} - \bar{y}_{i\cdot})^2
\end{align*}

\section*{Problem 2}

\begin{enumerate}
	\item ~
		\begin{center}
		\begin{tabular}{c|cccc}
			요인 & 제곱함 & 자유도 & 평균제곱 & F 값 \\
			\hline
			인자 A & 476.85  & 3  & 158.95 & 2.531 \\
			잔차 E & 2009.92 & 32 & 62.81 \\
			\hline
			계& 2486.77 & 35 &
		\end{tabular}
		\end{center}
	\item $H_0: \alpha_i = 0, H_1: \text{at least one}\ \alpha_i \ne 0$
	\item $F \sim F(3, 32)$
\end{enumerate}

\section*{Problem 6}

\begin{gather*}
	SS_T = 132.0, SS_A = 7.0, SS_B = 10.0, SS_E = 115.0 \\
	MS_A = 1.4, MS_B = 2.5, MS_E = 9.583 \\
	H_0: \alpha_i = 0, H_1: \text{at least one}\ \alpha_i \ne 0 \\
	f_1 = MS_A / MS_E = 0.146 \\
	f_1 \le F_{0.05}(4, 12) = 3.259 \\
	H_0: \beta_i = 0, H_1: \text{at least one}\ \beta_i \ne 0 \\
	f_2 = MS_B / MS_E = 0.261 \\
	f_2 \le F_{0.05}(3, 12) = 3.49
\end{gather*}

In both cases, we fail to reject the null hypothesis. In both cases, there is
not enough evidence that either treatment affects the residual impurities.

\section*{Problem 8}

\begin{center}
\begin{tabular}{c|ccccc}
	요인 & 제곱합 & 자유도 & 평균제곱 & F 값 & 유의확률 \\
	\hline
	A        & 3.337 & 3 & 1.112 & 10.111 & 0.001 \\
	B        & 0.166 & 2 & 0.083 &  0.754 & 0.492 \\
	교호작용 & 2.571 & 6 & 0.428 &  3.895 & 0.022 \\
	잔차     &  1.32 & 12 & 0.11 \\
	\hline
	계       & 7.393 & 23
\end{tabular}
\end{center}
\begin{gather*}
	H_0: \alpha_i = 0, H_1: \text{at least one}\ \alpha_i \ne 0 \\
	f_1 \ge F_{0.05}(3, 12) = 3.49 \\
	H_0: \beta_i = 0, H_1: \text{at least one}\ \beta_i \ne 0 \\
	f_2 \le F_{0.05}(2, 12) = 3.885 \\
	H_0: \gamma_i = 0, H_1: \text{at least one}\ \gamma_i \ne 0 \\
	f_2 \ge F_{0.05}(6, 12) = 2.996
\end{gather*}

The temperature, as well as the combination of the temperature and 종매량 show
evidence of being significant. There is not enough evidence to show that 종메량
alone affects hardness.

\end{document}
